\documentclass{article}
% Comment the following line to NOT allow the usage of umlauts
\usepackage{amsmath}
\usepackage[utf8]{inputenc}
% Uncomment the following line to allow the usage of graphics (.png, .jpg)
%\usepackage{graphicx}

% Start the document
\begin{document}

\title{Proposal Kelas Olimpiade Matematika}
\author{Arief Anbiya}
\maketitle

% Create a new 1st level heading
\section{Pendahuluan}

\subsection{Latar Belakang dan Tujuan}

\section{Struktur Materi}

\begin{itemize}

\item \textbf{Mental Toughness and Creativity}

Di dunia olahraga fisik, seorang atlet perlu otot yang kuat/mampu untuk melakukan berbagai gerakan, sekedar tahu teori gerakannya saja tidak cukup.

Sama halnya dengan olimpiade matematika, dalam menghadapi sebuah \textit{problem} kita perlu stamina atau daya pikir yang kuat untuk menemukan solusinya. Seringkali juga solusi masalah olimpiade matematika harus ditemukan secara kreatif.

Untuk melatih \textit{mental toughness}, ada beberapa sesi latihan yang mudah dimengerti dan dilakukan untuk pemula, seperti contoh berikut: Untuk menghitung $235^{2}$, kita dapat melakukannya lebih efisien dengan menggunakan sifat $(a-b)(a+b)= a^{2}-b^{2}$. 
$$(235-35)(235+35) = (200)(270) = 235^{2} - 35^{2} \implies 54000 + 35^{2} = 235^{2}$$
jadi kita hanya perlu menghitung $35^{2}$. Untuk pemula, siswa dapat diberikan latihan seperti ini untuk menghitung $101^{2}, ..., 500^{2}$ untuk melatih stamina mental mereka. 

Contoh lain, siswa juga dapat melatih kemampuan fokus mereka dengan latihan tes koran.

\item \textbf{Problem Solving Strategies}
Beberapa strategi untuk pemecahan masalah: mencari data, mencari pola, eksperimen berbagai cara/metode, berspekulasi, dan induksi matematika.


\item \textbf{Focus Topics: Algebra, Combinatorics}

Topik yang diuji di kompetisi matematika (OSN, IMO, dll) adalah Aljabar, Kombinatorika, Geometri, dan Teori Bilangan. 

Untuk kelas ini, pembelajaran akam difokuskan ke topik Aljabar dan Kombinatorika. Pembelajaran Geometri dan Teori Bilangan tetap ada namun lebih sedikit.

 Soal-soal akan diambil dari sumber berikut:

\begin{itemize}
\item "The Art and Craft of Problem Solving", Paul Zeitz
\item Soal-soal OSN Matematika Tingkat SMP (untuk pemula dan menengah) dan SMA. 
\item Sedikit soal dari IMO (\textit{International Mathematics Olympiad}).
\item Soal-soal dari website
math.stackexchange.com

\item Sumber lain jika ada.
\end{itemize}

\end{itemize}

\section{Beberapa Contoh Soal dan Pembahasan}


\section{Kelas}

Satu kali pertemuan adalah 3 jam. Pada tiap pertemuan dapat berupa: Latihan Soal dan Pembahasan, Kuliah/Presentasi Materi, atau Belajar Cara Membaca Buku atau Solusi Matematika.

\begin{itemize}

\item Jadwal: hari Jumat sore/malam, atau hari Sabtu pagi/siang. 
\item Biaya per sesi: 250,000.

\end{itemize}

\section{Biografi Pengajar}





\end{document}
